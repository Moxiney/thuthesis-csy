% !TeX root = ../thuthesis-caishiyu.tex

\chapter{引言}

\section{研究背景}

\subsection{非易失性内存}

% 非易失性内存
非易失性内存(non-volatile memory, NVM)是一种新的存储介质,其结合了内存 (DRAM)和磁盘的优点。NVM 具有低延迟,持久化,字节寻址以及大容量等特点。
具体地说,NVM 的读写延迟与内存 (DRAM) 处于一个数量级。并且 NVM 的数据是持久化的,也就是说当 NVM 断电之后,NVM 上的数据不会消失。
同时 NVM 和 DRAM 一样,允许上层应用以字节粒度来访问数据,而磁盘等介质只能以块粒度访问数据。最后,NVM 的存储容量相较于 DRAM 大很多。
\todo{找一下 NVM 的详细说明。}

目前为止,市面上有两款商用化的 NVM 产品,分别是 NVMDIMM-N 以及 Intel Optane DC PMM(后简称 Optane)。\todo{增加这两个产品的说明。}

% 非易失性内存对于数据库的影响
\subsection{数据库管理系统对于 NVM 的应用}

\todo{先介绍一下数据库基本概念}
这是因为传统的数据库使用 DRAM 和磁盘两层存储架构。
数据库系统使用 DRAM 作为一个高速的数据缓存介质,用来提高系统性能。而 DRAM 上的数据是易失性的,因此为了保证事务所造成的影响是持久化的,
数据库管理系统需要使用磁盘这一类的非易失性存储介质。然而内存和磁盘之类的访问性能差异十分巨大,\todo{增加数据说明。},数据库管理系统需要管理和同步两种介质之间的数据,
增加了额外的开销。

NVM 的出现及商用化激起了许多研究者的兴趣,尤其是在数据库领域。NVM 结合内存和磁盘的优势,因此对于数据库而言,单层的存储架构是可能的。
单层的存储架构可以降低复杂的数据管理开销以及数据同步开销,同时 NVM 和内存的访问速度处于一个数量级。因此理论上以 NVM 为主体的单层存储结构能够提高数据库管理系统的性能。

现有的 NVM 数据库工作着重于 NVM 来提高系统的性能。\todo{介绍几种使用 NVM 的方法,着重找一个重点介绍}


\subsection{基于 NVM 的数据库的垃圾回收和数据恢复}

\section{研究难点}

\section{研究内容及研究贡献}

\section{文章的组织结构}

