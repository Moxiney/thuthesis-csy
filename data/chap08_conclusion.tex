% !TeX root = ../thuthesis-caishiyu.tex

\chapter{总结与展望}

\section{本文工作总结}

\todo{重新编写此部分内容}

本文研究分析 NVM 的硬件特性,为 N2DB 设计了垃圾回收机制以及数据恢复机制。为了克服上述挑战, 本文采用了三个主要的设计方法:

\begin{enumerate}
    \item 树状数据结构:本文将 NVM 上的所有数据结构设计成树状结构,并且将其根指针持久化在 NVM 空间的固定区域。当系统容灾恢复时,能够通过固定的根指针入口按层次遍历找到 NVM 上所有的数据结构。同时根据根指针中记录的类型信息,系统可以正确解读数据结构中的数据。
    \item 垃圾回收机制与数据恢复机制协同工作:为了实现无日志的数据恢复机制,本文采用与无日志分配器类似的垃圾回收机制协同的数据恢复策略。当系统数据恢复时,数据恢复机制无法根据持久化的信息迅速判断内存泄漏的空间。因此系统会提前提供服务。对于仍处于内存泄漏状态的空间,系统会创建一个扫描线程,扫描所有的正在被使用的空间。该线程会将扫描的信息传递给垃圾回收组件,而垃圾回收组件负责根据扫描信息异步地回收内存泄漏的空间。
    \item 可见性判断:系统因为故障宕机时,NVM 空间上将会存在未提交事务的影响,这些影响会造成 NVM 空间上的部分数据结构的不一致。本文设计了两种重要的数据结构的可见性判断,系统可以根据可见性判断无视掉不一致的数据结构。因此系统在数据恢复时,仅需要根据可见性判断无视掉不一致的数据结构,就能避免未提交事务的影响。同时系统能够尽可能少地修改 NVM 上的数据,加快了系统恢复相应的速度。
\end{enumerate}

本文基于上述方法,提出了适配 N2DB 的垃圾回收机制,并且提出了第一个不依赖日志的高速的数据恢复机制。本文在 Intel Optane DC PMM 环境下测试了了两种机制的效果。实验表明,垃圾回收机制能够在至多降低 $10\%$ 的运行时性能的前提下,帮助系统节约至多 $67\%$ 的存储空间。同时与 InnoDB 的恢复性能对比实验表明,本文所提出的无日志数据恢复机制的恢复时间十分迅速,1.5 GB 左右的记录数据仅需要 0.85s 就能恢复成功。同时该恢复时间比基于写前日志的恢复时间低至多三个数量级。并且该数据恢复机制所使用的存储空间仅仅是 InnoDB 的一半。

\section{不足之处以和未来研究方向}

本文研究分析 NVM 的硬件特性,为 N2DB 设计了垃圾回收机制以及数据恢复机制。为了克服上述挑战, 本文采用了三个主要的设计方法:

\begin{enumerate}
    \item 树状数据结构:本文将 NVM 上的所有数据结构设计成树状结构,并且将其根指针持久化在 NVM 空间的固定区域。当系统容灾恢复时,能够通过固定的根指针入口按层次遍历找到 NVM 上所有的数据结构。同时根据根指针中记录的类型信息,系统可以正确解读数据结构中的数据。
    \item 垃圾回收机制与数据恢复机制协同工作:为了实现无日志的数据恢复机制,本文采用与无日志分配器类似的垃圾回收机制协同的数据恢复策略。当系统数据恢复时,数据恢复机制无法根据持久化的信息迅速判断内存泄漏的空间。因此系统会提前提供服务。对于仍处于内存泄漏状态的空间,系统会创建一个扫描线程,扫描所有的正在被使用的空间。该线程会将扫描的信息传递给垃圾回收组件,而垃圾回收组件负责根据扫描信息异步地回收内存泄漏的空间。
    \item 可见性判断:系统因为故障宕机时,NVM 空间上将会存在未提交事务的影响,这些影响会造成 NVM 空间上的部分数据结构的不一致。本文设计了两种重要的数据结构的可见性判断,系统可以根据可见性判断无视掉不一致的数据结构。因此系统在数据恢复时,仅需要根据可见性判断无视掉不一致的数据结构,就能避免未提交事务的影响。同时系统能够尽可能少地修改 NVM 上的数据,加快了系统恢复相应的速度。
\end{enumerate}

