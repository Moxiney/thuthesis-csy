% !TeX root = ../thuthesis-caishiyu.tex

\chapter{总结与展望}

\section{本文工作总结}


数据库管理系统是广泛使用的计算机软件。传统的数据库管理系统使用内存-持久化存储介质的双层存储架构,一层作为高速的数据易失的缓存,一层作为大容量的数据存储介质。

非易失性内存是结合内存以及固态硬盘的优点。其作为一个新兴的存储介质给数据库管理系统的研究提供了新的方向。
NVM 优异的硬件特性允许 DBMS 采用单层的存储架构,避免了运行时的复杂的数据管理开销以及频繁的数据同步,进而提高了系统的性能,降低了单个事务的延迟。
N2DB 是第一个零拷贝,无日志的数据库管理系统。
无日志的特性将是 NVM 数据库的一大趋势。

垃圾回收以及数据恢复是基于 NVM 的数据库管理系统中重要的组件。
垃圾回收机制负责回收 NVM 介质上冗余的数据,以最大化利用 NVM 存储空间。
数据恢复机制让 DBMS 从故障中恢复到可工作的状态。
数据恢复机制需要保证系统满足 ACID 四个特性。
然而无日志的前提给数据库的垃圾回收机制以及数据恢复机制带来了挑战。
并且传统的垃圾回收机制以及数据恢复机制难以迁移到 NVM 数据库上。
因此本文在 N2DB 中设计并实现了不基于日志的垃圾回收机制以及数据恢复机制。

本文为了克服 NVM 上编程的挑战,结合 NVM 相关工作使用了 3 个设计方法,树状数据结构,懒惰垃圾回收以及可见性判断。
本文在 Intel Optane DC PMM 环境下测试了两种机制的效果。实验表明,垃圾回收机制能够在至多降低 $10\%$ 的运行时性能的前提下,帮助系统节约至多 $67\%$ 的存储空间。同时与 InnoDB 的恢复性能对比实验表明,本文所提出的无日志数据恢复机制的恢复时间比基于写前日志的恢复时间低至多三个数量级。并且该数据恢复机制节约了一半的存储空间。

本文提出的垃圾回收机制和数据恢复机制可以为其他无日志的数据库管理系统的设计提供参考。
其他的 NVM 数据管理系统只要将事务状态持久化并且将数据结构设计成树状结构,就能采用本文的垃圾回收机制以及数据恢复机制,进而实现高速的容灾恢复。


\section{不足之处和未来研究方向}

截止目前,本文所实现的垃圾机制以及数据恢复机制主要有两个不足之处。
一是垃圾回收机制的开销大,并且没有考虑事务长时间运行对于垃圾回收的阻塞问题。
二是数据恢复的恢复时间还没达到理论上的极限。
现有工作仍需要扫描所有表格的 head,该扫描工作占据了系统恢复的大部分开销。

这两个不足之处是本文未来的工作方向。N2DB 可以采用 Hyper 类似的存储结构,将事务的前像存储在连续的物理空间中,并且一次性回收。采用这样的存储架构理论上能够降低垃圾回收的开销,但是需要针对四种可回收的对象具体分析。
系统需要设计更多的版本相关的操作才能在不影响长时间运行的版本的前提下回收其他可回收空间。
另外 N2DB 可以通过实现持久化索引来降低重启之后的开销。理论上可以通过设计索引的节点的可见性来实现即刻恢复的数据恢复机制,但是持久化索引相对于内存中的索引的读性能较低,可能会影响事务的吞吐率。
