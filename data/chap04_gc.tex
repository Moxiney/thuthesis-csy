% !TeX root = ../thuthesis-caishiyu.tex

\chapter{垃圾回收机制设计}

\section{本章概述}

\todo{将本章节改成设计目标}

数据库在运行的过程中可能会删除记录,表格以及其他数据,而这些数据会占据实际的存储空间。因此系统定期对这些冗余的数据进行回收,以提高存储空间的利用率。对于采用 MVCC 的数据库而言,垃圾回收更加重要。这是因为 MVCC 的事务在执行的过程中,会创造出很多的版本以降低读写冲突的可能性。但是过场的版本链会降低事务执行的性能,因此 MVCC 数据库也需要定期回收冗余的版本。总而言之,数据库回收策略的对象是冗余的数据,具体地来说就是对于其他运行时的事务不可见的版本,行,表格。

NVM 数据库的垃圾回收策略有以下难点:

\begin{enumerate}
    \item 由于数据库是多线程并发的,因而在垃圾回收的设计时要当保证其他任何运行的事务及之后的事务不会访问到正在被回收的空间。
    \item 垃圾回收的对象是 NVM 上的数据结构,因此要保证对于上面数据的修改能够保证数据结构的崩溃一致性,降低系统故障恢复的开销。
\end{enumerate}

在我们的系统实现中,垃圾回收的场景主要可以分为两个类别。一是在运行时的垃圾回收,用于回收逻辑上不可见,物理上不会被访问到的数据结构。二是在系统重启恢复时的垃圾回收,用于回收未被使用的空间。

两种情景下的垃圾回收机制都与数据库的并发控制算法息息相关。
因此章节~\ref{sec:mvcc} 先介绍了存储引擎所使用的并发控制算法。该章节会着重介绍在保证数据结构的崩溃一致性的前提下指令顺序的设计。
之后章节~\ref{sec:space} 会从宏观层面介绍在运行过程以及数据恢复过程中不可见数据的判断方式。
章节~\ref{sec:time} 会解释垃圾回收的正确的时机。
最后章节~\ref{sec:implement} 详细地介绍垃圾回收的流程,以及具体的相关数据结构设计,并且解释了垃圾回收机制如何保证多线程编程的正确性的。

\section{多版本并发控制算法}
\label{sec:mvcc}
N2DB 采用了多版本并发控制算法。
本章节首先介绍事务开始的流程,之后介绍基于快照的版本可见性判断,接下来介绍了事务在运行时访问版本链的流程,以及读、插入、更新以及删除这 4 个基本操作的流程,最后简单提及事务提交和中止的流程。

\begin{figure}[ht]
    \centering
    \includegraphics[width=0.6\linewidth]{example-image-a.pdf}
    \caption{版本可见性的判断流程图}
    \label{fig:version-visibility}
\end{figure}

事务开始时会先向申请一个全局唯一的事务 ID。同时事务会获得一个快照 (snapshot)。快照逻辑上可以视为已经结束的事务的集合。已经结束的事务既包括提交的事务也包括中止的事务。一个事务的快照决定了该事务能看到的事务的影响的范围,同时在事务运行时,快照是不会改变的。

一条记录可能有多个版本。事务必须基于快照和其他信息才能判断这些版本中哪一个才是对该事务可见的 (visible) 版本,称之为版本可见性判断。一个记录对于一个事务而言,至多有一个可见的版本。图~\ref{fig:version-visibility} 中展示事务的版本可见性判断策略。而别的事务提交还是中止的判断方法则是通过访问事务状态数组得到对应的事务状态来判断。版本的可见性判断策略可以分为主要两步:(1)该版本的创建事务已经结束且提交了,否则该版本不可见 (invisible)。(2)该版本尚未被其他事务废弃,即版本的 xmax 为非法值,则说明该版本是可见的。(3) 该版本被其他事务废弃,然而废弃事务尚未提交也未结束,则该版本是可见的。

\begin{algorithm}[ht]
    \caption{事务访问版本链的方法 $access\_version$}
    \label{alg:traverse_version_chain}
    \KwIn{The accessed head, $h$\ and the snapshot of the accesser.}
    \KwOut{Visible version, $v$}
    \BlankLine
    \If{ $h$ is empty or $h$ is already deleted}{
        return $nullptr$;
    }

    Set $v$ to the newest version of $h$;

    \While{ $v$ is not $nullptr$ and $v$ is not visible according to snapshot}{
        Set $v$ to the older version of $v$;
    }

    return $v$;

\end{algorithm}

算法~\ref{alg:traverse_version_chain} 中为访问版本链的流程。
事务在访问版本链的过程中会先判断 head 是否合法,即 head 的最新版本不为空且该记录尚未被删除。之后事务会从新到旧遍历所有的版本,直到找到可见的版本。如果找不到可见的版本则会返回空指针。


事务在运行时有四种基本操作。同时事务在运行时会将访问过的版本分别记录在四个集合,分别为 read\_set,update\_set,insert\_set 和 remove\_item。

\textbf{读操作:}算法~\ref{alg:transaction-read} 为读操作的流程。对于一个指定的记录的 ID,事务首先通过表格的 head\_pages 定位其对应的 head。
之后从该 head 开始,遍历地访问版本链,直到找到对于该事务可见的版本,否则则说明对于该事务而言,该记录不存在。

\begin{algorithm}[h]
    \caption{事务的读操作 $read$}
    \label{alg:transaction-read}
    \KwIn{The record id, $rid$}
    \KwOut{The visible version, $rv$}
    \BlankLine
    Get the record head $rh$ using the record id, $rid$;

    return $access\_version(rh, snapshot)$;

\end{algorithm}

\textbf{插入操作:}算法~\ref{alg:transaction-insert} 为插入操作的流程。首先表格分配一个新的 head 给事务。接下来表格再分配一个新的 version。事务将记录的数据写入写的版本里,并且写入其他的元数据,比如说事务 ID。接下来事务使用 fence 指令以保证在写入 head 之前,数据及元数据已经写入 version 中。最后事务修改 head 的元数据以及修改 head 的 newest\_version 的值。最后事务使用一个 fence 指令保证事务结束该操作时,所有的 nvm 写入操作都已完成。

\begin{algorithm}[h]
    \caption{事务的读操作 $insert$}
    \label{alg:transaction-insert}
    \KwIn{Record data}
    \KwOut{Success or not}
    \BlankLine
    Table allocate a new head, $rh$;

    Table allocate a new version, $rv$;

    Write data into the new version, $rv$;

    \textbf{Fence};

    Modify the newest version of $rh$ to $rv$;

    \textbf{Fence};

    return $true$;

\end{algorithm}

\textbf{更新操作:}算法~\ref{alg:transaction-update} 为更新操作的流程。事务首先从根据记录 ID 找到对应的 head,之后根据访问版本链的策略得到找到对应的 version。接下来事务需要判断有没有更新冲突,比如有一个并发的事务在修改同一个记录。之后表格分配新的 version,事务将记录数据以及其他元数据写入到该 version 中。之后事务修改可见事务的 xmax ,之后使用 fence 指令保证持久化顺序。
最后事务修改 head 的 newest\_version 为新分配的 version,之后使用 fence 指令。

\begin{algorithm}[h]
    \caption{事务的读操作 $insert$}
    \label{alg:transaction-update}
    \KwIn{Record data and the record id, $rid$}
    \KwOut{Success or not}
    \BlankLine
    Get the record head $rh$ using the record id, $rid$;

    Find the visible version $rv$ using $rh$;

    \If{there is update conflict}{
        return $false$;
    }

    Table allocate a new version, $new\_rv$;

    Write data into the new version, $new\_rv$;

    Modify the $xmax$ of $rv$;

    \textbf{Fence};

    Modify the newest version of $rh$ to $new\_rv$;

    \textbf{Fence};

    return $true$;

\end{algorithm}


\textbf{删除操作:}算法~\ref{alg:transaction-delete} 为删除操作的流程。事务首先根据记录 ID 在表格中定位对应的 head 的地址。之后事务找到对于该事务的可见版本 version。如果该版本为最新版本时,则意味着没有事务冲突,接下来事务更新 head 的 remove\_tx 为该事务的事务 ID,之后使用 fence 指令保证持久化顺序。如果该版本不是最新版本,则该事务删除操作失败。

\begin{algorithm}[h]
    \caption{事务的读操作 $delete$}
    \label{alg:transaction-delete}
    \KwIn{The record id, $rid$}
    \KwOut{Success or not}
    \BlankLine
    Get the record head $rh$ using the record id, $rid$;

    Find the version $rv$ using $rh$;

    \eIf{$rv$ is the newset version of $rh$}{
        Modify the $xmax$ of $rv$;

        Modify the $remove\_tx$ of $rh$;

        \textbf{Fence};

        return $true$;
    }{
        return $false$;
    }

\end{algorithm}

\textbf{事务的提交和中止流程:} 首先会通过事务状态数组的 $set\_status$ 接口将该事务的状态设置为对应状态。之后事务会先记录当前已分配的最大事务 ID,$max\_seen\_tid$。
接下来事务根据自身访问过的记录以及在记录上实施的操作,判断出在该事务结束后可回收的空间。最后事务将可回收的空间信息与 $max\_seen\_tid$ 一同传递给运行时的垃圾回收组件。
垃圾回收组件会根据这些信息决定回收的时机。


\section{垃圾回收的对象}
\label{sec:space}

在本章节中,我们会具体介绍运行时以及恢复时可回收的空间的判断方法。

\subsection{运行时垃圾回收的对象}

事务在运行过程中会根据访问的版本信息标记可以回收的空间。当事务提交或者中止后会将可回收的空间信息传递给垃圾回收机制,由垃圾回收机制决定的回收的时机。运行时的可回收的空间根据事务提交或者中止可以分为两大种情况。

\begin{figure}
    \centering
    \includegraphics[width=0.6\linewidth]{example-image-a.pdf}
    \caption{提交事务所标记的可回收空间}
    \label{fig:space-commit}
\end{figure}

\begin{figure}
    \centering
    \includegraphics[width=0.6\linewidth]{example-image-a.pdf}
    \caption{提交事务所标记的可回收空间}
    \label{fig:space-commit2}
\end{figure}

首先是提交事务的所标记的可回收空间。如图~\ref{fig:space-commit} 所示,有两种情况:(1) 当该事务进行更新操作时,会创建一个新的版本。因此当该事务提交之后,比所创建的版本更早的版本可被回收。(2) 事务删除某一条记录并且提交后,该记录所对应的整个版本链也可以被回收。 当该提交事务的影响被所有事务以及后续事务可见时,该事务所标记的空间就是不可见的。因此这些空间可以被回收。\todo{举个例子:事务 10 更新产生了一个新版本,之前的版本blablabla}

\begin{figure}
    \centering
    \includegraphics[width=0.6\linewidth]{example-image-a.pdf}
    \caption{中止事务所标记的可回收空间}
    \label{fig:space-abort}
\end{figure}

\begin{figure}
    \centering
    \includegraphics[width=0.6\linewidth]{example-image-a.pdf}
    \caption{中止事务所标记的可回收空间}
    \label{fig:space-abort2}
\end{figure}

其次是中止事务的所标记的可回收空间。如图~\ref{fig:space-abort}所示,同样有两种情况:(1) 一个事务进行了插入操作,当事务中止之后,因为插入操作所新增的版本链是可以被回收的。(2) 当事务更新了一个记录并且中止之后,所创建的新版本也是不可见的,因此也是可回收的。\todo{举个例子,事务 10 blabla}

\subsection{恢复时垃圾回收的对象}

系统在数据恢复的过程中同样也会标记可回收的空间。恢复时的可回收空间可以分为两类:(1)所有的表格中未被分配的数据结构(head 和 version)均是可回收的。(2)上一次运行中的所有事务对于这次运行而言均是结束的。因此版本链中所有不可见的 version 以及不可见的 head 都是可回收的。

\section{垃圾回收的时机}
\label{sec:time}

本章节将会介绍运行时的四种可回收的空间的回收时机,以及恢复时的可回收空间的回收时机。

\subsection{运行时的可回收空间的回收时机}

当事务提交并且将可回收的空间信息传递给垃圾回收机制之后。垃圾回收回收机制会在该提交事务的影响对于所有活跃的事务可见,并且对于所有后续的事务也可见的情况下,对所标记的空间进行回收。因为当该事务对于全局及后续的事务可见后,所有的后续事务逻辑上永远不会访问到比该事务更新所创建的版本更早的版本。同时该事务删除记录的影响对全局可见意味着所有全局事务逻辑上不会访问到被删除的记录。

中止事务的所标记的空间情况相对复杂。中止事务的影响是对于全局事务永远都不可见的。因此全局事务不可能访问到中止事务所插入的新的记录,所以插入的新的版本链可以被立刻回收。中止事务更新所创建的新版本会根据版本可见性原则被所有事务无视。然而并发的事务在访问版本链的过程中有持有该版本的地址。因此中止事务所创造的新版本不能立刻回收,需要从版本链上断开,如图~\ref{fig:insert-abort} 所示。当并发的所有事务均结束后,垃圾回收组件才可能安全地回收该版本,同时不会对其他事务的访问版本链的行为产生影响。

\todo{需不需要举个例子:找一找之前分析 gc 设计的那几种冲突情况}

\begin{figure}
    \centering
    \includegraphics[width=0.6\linewidth]{example-image-a.pdf}
    \caption{事务在中止时对于更新创建的新版本进行的断链操作。}
    \label{fig:insert-abort}
\end{figure}

除了中止事务插入的版本链可以立刻被回收外,其他三种情况均需要延后一段时间后才能回收。因此实际实现中采用了一个计数器来帮助垃圾回收线程判断回收时机。当事务提交或者中止时,事务会访问全局的事务 id 的计数器,记录一个最大的未被分配的事务 id,记为 $max\_seen\_tid$。这三种可回收空间的可回收时机为活跃的最小事务 id 大于 $max\_seen\_tid$。因为当该条件满足时,提交事务的影响会对于全局及后续事务可见,中止事务的并发的事务也已经结束了。

\subsection{恢复时的可回收空间的回收时机}

恢复时的可回收空间一共有几种情况,分别有不同的回收时机:(1)对于所有不可见的 head 可以被立刻回收。(2) 版本链中所有不可见的 version 也可以被立刻回收的,也可以有运行时的垃圾回收机制回收。(3)所有未被分配的版本,也就是不在版本链中的 version,需要等到数据恢复机制的后台扫描线程扫描了所有版本链之后才能回收。


\section{系统工程实现}
\label{sec:implement}

本章节会着重介绍在工程实现中的垃圾回收机制设计,包括垃圾回收的频率,粒度,以及对并发和数据结构的崩溃一致性的设计和考量。章节~\ref{ssec:gc-metadata} 会先介绍我们垃圾回收调度器的数据结构定义。然后章节~\ref{ssec:gc-implement} 中展示了事务提交和中止中与垃圾回收相关的操作。最后章节~\ref{ssec:gc-implement} 解释详细的垃圾回收流程.

\subsection{垃圾回收调度器的数据结构}
\label{ssec:gc-metadata}

% struct GCItem {
%     storage::Table *tbl;
%     ID_TYPE xmax;
%     ID_TYPE max_seen\_tid;
%     storage::RowVersion *rv;
%     storage::RowHead *rh;
%     ID_TYPE row_id;
%     GCTS_TYPE gc_ts;
% };

可回收的版本会被事务封装成 GCItem。GCItem 的数据结构展示在图~\ref{fig:gc-item} 中。
tbl 记录了可回收的版本所属于的表格。xmax 用于记录废弃该版本的事务 ID,通常为标记该版本的事务的事务 ID。
max\_seen\_tid 则为封装 GCItem 在封装时所看到的最大的已分配的事务 ID,用于给垃圾回收调度器判断回收时机。
rv 代表了需要回收的版本的指针。当一个版本链上有复数个连续的版本需要回收时,事务通常只会记录这些版本中最新的版本。
rh 则为可回收空间所对应的 head 的指针。row\_id 则为该记录所对应的 ID。最后 gc\_ts 记录了 head 的回收次数,用于防止并发的垃圾回收调度器对同一行进行重复回收。

\begin{figure}
    \centering
    \includegraphics[width=0.6\linewidth]{example-image-a.pdf}
    \caption{GCItem 的数据结构}
    \label{fig:gc-item}
\end{figure}


事务会将 GCItem 传递给同线程的垃圾回收调度器(GCScheduler),后者负责具体的垃圾回收操作。垃圾回收调度器提供了 schedule 接口,
事务可以使用该接口将 GCItem 添加到调度器中的 gc\_items。
垃圾回收调度器的数据结构的成员变量如下:
\begin{itemize}
    \item 回收队列(gc\_queue):回收队列是一个先进先出(FIFO)的队列,其中每一个元素均为 GCItem。当本线程的事务结束之后,会通过 schedule 接口将封装成 GCItem 的可回收空间添加到该队列中。
    \item 回收阈值(gc\_threshold):回收阈值是一个启动时就已经设定好的正整数,其用于控制垃圾回收的频率。当回收队列的长度大于回收阈值时,调度器就会开始回收 GCItem 以减少队列的长度。如果回收阈值为 0 时,则意味着每次事务结束时均会触发垃圾回收,系统中的冗余空间会处于一个极低的水平。但是系统频繁地调用垃圾回收会降低系统的性能。在我们的实现中,回收阈值通常为 100。
\end{itemize}



\subsection{事务的提交和中止的流程设计}
\label{ssec:commit-abort}

事务在提交和中止事务的流程中,不仅需要将可回收的空间信息封装 GCItem,同时也会需要进行一些额外操作,比如说回收中止事务插入的版本链和将中止事务更新的新版本移出版本链。本章节将介绍事务提交和中止流程的设计思路以及具体实现。

\begin{algorithm}[h]
    \caption{事务提交的流程}
    \label{alg:commit}
    \BlankLine
    Set transaction status to COMMITTED;

    \ForAll{ accessed versions $rv$ in update\_set}{
        Wrap the previous version of $rv$ into a GCItem; \\
        Append the GCItem into the local queue, gc\_queue;\\
    }

    \ForAll{ accessed version $rv$ in remove\_set}{
        Wrap $rv$ into a GCItem; \\
        Append the GCItem into the local queue, gc\_queue;\\
        Modify $remove\_tx$ of the head;\\
    }

    Record the max\_seen\_tid;

    \ForAll{
        item in gc\_queue
    }{
        Modify item using max\_seen\_tid; \\
        GCScheduler.schedule(item);

    }


\end{algorithm}

\textbf{提交流程:} 如算法~\ref{alg:commit} 所示,事务首先将通过事务状态数组的 set\_status 接口将自身状态设置为 COMMITTED。
接下来对于 update\_set 中的所有版本,将更新所创建的新版本的更老的版本和其他信息封装成 GCItem,添加到事务本地的队列中。
之后对于 remove\_set 中的鄋版本,事务将该记录的版本链的最新版本与其他信息封装成 GCItem,然后同样将其添加到本地队列。同时事务需要设置对应 head 的 remove\_tx 为本事务的事务 ID。
在封装两种情况所对应的 GCItem 之后,事务读取全局的事务 ID 计数器,得到当前最大的未分配的事务 ID,$max\_seen\_tid$。最后事务将本地队列清空,使用 $max\_seen\_tid$ 更新所有 GCItem,之后事务将所有的 GCItem 通过垃圾回收调度器的 schedule 接口传递给调度器。


\begin{algorithm}[h]
    \caption{事务中止的流程}
    \label{alg:abort}
    \BlankLine
    Set transaction status to ABORTED;

    \ForAll{ accessed versions $rv$ in insert\_set}{
        Table reclaims $rv$;
        Table reclaims head of $rv$;
    }

    \ForAll{ accessed version $rv$ in update\_set}{
        Find the head $rh$; \\
        Compare\_and\_swqp($rh.newset\_version$, $rv$, $rv.prev\_version$); \\
        Wrap $rv$ into a GCItem; \\
        Append the GCItem into the local queue, gc\_queue;\\
    }

    Record the max\_seen\_tid;

    \ForAll{
        item in gc\_queue
    }{
        Modify item using max\_seen\_tid; \\
        GCScheduler.schedule(item);

    }


\end{algorithm}

\textbf{中止流程:} 事务中止的流程与事务提交相近,但是更为复杂。首先事务先设置事务状态数组中的自身的状态为 ABORTED。接下来 insert\_set 中的所有版本可以立刻被表格回收。当版本被回收之后,其对应的 head 也可以被表格回收。
之后对于 update\_set 中的所有版本,事务使用一个 CAS 指令原子化修改版本的 head 的 newest\_version 指针,修改成功的话则说明该版本被成功地从版本链上断开。然后事务将 update\_set 中的版本与其他信息封装成 GCItem,将其添加到本地队列中。最后事务记录记录最大的未分配的事务 ID,更新本地队列中所有的 GCItem,最后使用垃圾回收调度器的 schedule 接口传递所有的 GCItem。

中止流程中有两个设计要点:

首先是记录最大的未分配的事务 ID 必须在更新所创建的版本断链之后。\todo{画图}假设两者顺序对调,以图~\ref{fig:update-abort-read} 为例,事务 1 中止时记录了最大未分配的事务 ID 为 2。而当事务 1 记录完之后,事务 2 开始了并且读到同样的一条记录,此时事务 2 可能会持久被标记的版本的指针。而如果事务 1 结束了中止流程并且将所有的 GCItem 传递给垃圾回收调度器,则垃圾回收调度器会认为当前事务 ID 已经大于版本所对应的最大的未分配的事务 ID,因此触发了回收。因此事务 2 可能会读到不一致的数据。

\begin{figure}
    \centering
    \includegraphics[width=0.6\linewidth]{example-image-a.pdf}
    \caption{事务访问版本连与事务中止流程的并发冲突}
    \label{fig:update-abort-read}
\end{figure}

其次是通过 CAS 指令修改 head 的 newest\_version 可以防止并发事务之间的冲突,同时可以保证对应版本断链。
假设使用原子性写来修改 newest\_version,以图~\ref{fig:update-abort-update} 为例,当事务 1 中止之后,事务 2 更新同一条记录。之后事务 2 创建新版本。事务 1 和事务 2 都需要修改 head 指针,前者需要将指针修改为对应版本的后一个版本,后者则要把指针修改为新分配的版本。因此如果事务 2 原子性写在前,而事务 1 原子写在后,则事务 2 所创建的新版本就不在版本链上。因此这种情况造成了事务 2 的更新丢失。而使用 CAS 指令可以防止该问题,同时也不引入锁的开销。

\begin{figure}
    \centering
    \includegraphics[width=0.6\linewidth]{example-image-a.pdf}
    \caption{事务更新与事务中止流程的并发冲突}
    \label{fig:update-abort-update}
\end{figure}



\subsection{垃圾回收的流程设计}
\label{ssec:gc-implement}

事务调用 schedule 接口会将 GCItem 添加到垃圾回收调度器的 gc\_items 队列中。当队列长度达到阈值 gc\_threshold 之后,如果满足条件则会触发 do\_gc 进行实际的回收工作。
do\_gc 的具体流程如算法~\ref{alg:do_gc} 所示。

首先垃圾回收调度器获取全局活跃的最小事务 ID,记为 ma\_tid。当 gc\_items 的中的
GCItem 满足其 max\_tid 小于等于 ma\_tid 时,则调度器可以回收该 GCItem 对应的空间。

接下来调度器需要判断 GCItem 的类型。在章节~\ref{sec:space} 中,需要延后回收的空间有三种,一是提交事务的更新操作对应的空间,二是提交事务的删除操作所对应的空间,三是中止事务的更新操作所对应的空间。当调度器判断 GCItem 属于第三种情况时,则调度器会立即回收 GCItem 所对应的版本,并且结束回收工作。


由于每个线程均有一个垃圾回收调度器,一个版本链上的空间可能会被重复回收。
因此调度器需要根据 head 中的 gc\_ts 判断该 head 对应的版本链是否已经被回收过了。如果 head 已经被回收了,
调度器就直接结束该 GCItem 的回收工作。

调度器从 head 开始从新到旧访问版本链。在访问版本链之前,调度器对于该记录进行加锁以防止并发的调度器和事务对于该版本链进行指针操作。如果没有找到 GCItem 所对应的版本,则意味着该版本已经被其他调度器回收了,因此调度器释放锁并且结束回收工作。如果调度器找到对应的版本,则通过一个原子写修改对应版本的更新版本的后向指针,将 GCItem 所对应的版本及后面所有更老的版本从版本链上断开,之后调度器释放锁。

调度器从断开的版本链从新到旧依次回收所有的版本。

在版本回收工作结束之后,如果 head 之后已经没有版本,则意味着该行是被删除了。之后调度器自增 head 的 gc\_ts,以防止其他调度器重复回收该记录。
最后该 head 被回收。

至此调度器对一个 GCItem 的回收工作结束。调度器会接下来 gc\_items 中下一个 GCItem 满不满足回收条件,若满足则重复此过程。

\begin{algorithm}[h]
    \caption{垃圾回收调度器 do\_gc 的流程}
    \label{alg:do_gc}
    \KwIn{The transaction ID of minumum active transaction, $ma\_tid$}
    \BlankLine

    \While{ $gc\_items$ is not empty
    }{
        $item = gc\_items.front()$;\\

        \If{ item.max\_tid > ma\_tid}{
            return;
        }

        \If{
            $item.rv.max == item.rv.min$
        }{
            Table reclaims $item.rv$;\\
            return;
        }

        \If{$item.rh.gc\_ts != item.gc\_ts$} {
            return;
        }

        Lock the version chain;

        \For{
            version $v$ in the version chain
        }{
            \If{ $v == item.rv$
            }{
                Find the newer version of $item.rv$;

                Spilt the version chain on $item.rv$;
            }
        }

        Unlock the version chain;

        Table reclaims all version in remaining chain;

        \If{
            $item.rh.newset\_version == nullptr$
        }{
            Update $item.rh.gc\_ts$;

            Table reclaim $item.rh$;
        }

        gc\_items.pop();
    }


\end{algorithm}

\section{本章小结}

本章节从设计目标,并发控制,回收对象,回收时机这四个方面介绍了 N2DB 中的垃圾回收机制的设计思路,之后从系统实现角度详细地介绍垃圾回收的具体流程。

首先,垃圾回收机制的设计目标在于回收存储引擎中分配但未被使用的空间。由于 N2DB 是多版本的数据存储结构,垃圾回收机制还需要回收分配但未被使用的 head 以及 version 来重复利用存储空间,以此提高存储空间利用率。

其次,由于垃圾回收机制与事务的并发控制算法紧密耦合,章节~\ref{sec:mvcc} 中大致介绍了事务的运行时协议以及四个基本操作的流程。

接下来,章节~\ref{sec:space} 在并发控制算法的基础上介绍了垃圾回收机制对于可回收的空间的判断标准。对于运行时的垃圾回收机制而言,可回收的空间可以根据事务提交与否和操作类型分为四类情况。对于恢复时的垃圾回收机制而言,可回收的空间可以分为两种,分别是上一轮运行过程所遗留下来的未被回收的空间,和重启之后所有的未被分配和分配但未被使用的空间。

然后,章节~\ref{sec:time} 介绍了运行时的四种可回收空间以及恢复时的两种可回收空间的回收时机。运行时的四种可回收空间中有一个类型可以在事务结束时立刻回收,而剩下三种需要延后一段时间才能在不影响后续事务的运行的前提下回收。恢复时的可回收空间可以分为三种情况,不可见的 head 需要被立刻回收,而剩下两种情况可以延后。

最后,章节~\ref{sec:implement} 从系统的角度给出了运行时的垃圾回收机制的实现。每个线程都有一个负责垃圾回收的组件,事务执行结束后会将可回收的空间信息封装成 GCItem 传递给该组件。之后组件在满足条件的情况下回收对应的 GCItem。由于垃圾回收组件与别的线程的事务和垃圾回收组件是并发执行的,因此本章节着重解释了针对于重复回收,指针修改等并发冲突的解决方法。
