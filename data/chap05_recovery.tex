% !TeX root = ../thuthesis-caishiyu.tex

\chapter{数据恢复机制设计}
\label{chap:recovery}

\section{设计目标}

数据库存储引擎的故障类型可以分为几种,事务故障,系统故障以及硬件故障。事务故障意味着事务执行的过程中出现了错误,通常通过数据库管理系统的回滚操作来清除掉。而系统故障通常包括操作系统故障以及宕机故障,通常会导致数据库管理系统运行中断。而硬件故障意味着存储介质出现严重错误,系统能难通过软件层面避免此问题。

数据恢复机制是数据库系统为了从系统故障中恢复成可工作状态而设计的。当系统中断时,DRAM 中的所有数据丢失,系统必须根据持久化介质上保存的数据将系统还原到系统中断时的状态。
传统的数据库通常使用日志系统记录所有的事务操作信息,在重启之后根据日志系统重做检查点之后的事务的操作,回滚未提交的事务的操作。

N2DB 使用 NVM 作为主要存储介质,大部分数据在系统故障后仍然保存在 NVM 上。
同时 N2DB 在运行过程中是无日志的,因此给数据恢复带来了挑战。总而言之,N2DB 的数据恢复机制必须达到以下 3 个设计目标:

\textbf{重构页面分配信息:}N2DB 的主要数据均存储在 NVM 上,同时是按照页粒度分配且管理的。
然而分配器的分配是无日志的,因此 N2DB 需要在数据恢复阶段重构正确的页面分配信息。
系统在正确的分配信息的基础上才能回收已分配但未被使用的页面,进而防止内存泄漏问题。

\textbf{恢复数据结构:}系统需要在恢复服务之前得到所有数据结构的地址信息以及类型信息。
而页面分配信息仅能帮助系统得到页面的使用情况,在缺乏数据结构的元信息的情况,所使用的页面也仅是字节流。同时数据库的记录大小是不定长的,其根据表格模式而改变。数据库的所有数据结构及其元信息需要妥善地设计,系统才能根据元数据中的数据结构信息以及表格的模式信息获得所有表格中所有的记录数据。

\textbf{保证事务的原子性和持久化:}数据库管理系统在数据恢复的过程中还需要保证事务状态的原子性和持久化,
因此系统需要保证提交事务的影响的持久化以及中止事务的影响的消除。由于 N2DB 在并发控制算法的设计中保证了当事务提交时,所有的数据均已持久化到 NVM 上。因此数据恢复流程需要具有识别以及消除中止事务的影响的能力。

数据恢复只有在满足上述三个设计目标的前提下,才能确保系统在恢复后能够正确地提供服务。



\section{存储引擎的结构修改}

N2DB 为了在重启之后能够满足以上三个设计目标,需要对存储引擎进行重新设计,主要涉及两部分,分别是 NVM 分配器的结构修改以及数据库元信息区的结构设计。

\subsection{NVM 分配器的结构修改}
修改后的 NVM 分配器结构如图~\ref{fig:nvm-allocator} 所示。从图中可以看出,相对于之前的 NVM 分配器,新版的 NVM 分配器主要增加一个持久化指针数组。指针区域位于 NVM 存储空间的第二个页面,其中存放的是 NVM 上重要的数据结构的根指针。指针区域的数据结构定义如下:
\begin{itemize}
    \item 根指针数量(num\_root):指针数量记录了持久化根指针数组中已存储的根指针的数量。
    \item 持久化根指针数组(persist\_root\_array):持久化指针数组是一个定长的数组。其中每一个持久化根指针都是一个二元组,前者是地址,后者是数据结构类型。每一个常用的 NVM 的数据结构类型都映射到唯一的数字。该映射表硬编码在程序中。
\end{itemize}
指针区域提供了两个结构,分别是 persist\_root 以及 get\_root。前者负责将数组持久化,后者负责读取指定位置的根指针。

\begin{figure}[ht]
    \centering
    \includegraphics[width=0.6\linewidth]{example-image-a.pdf}
    \caption{修改后的 NVM 分配器的结构}
    \label{fig:nvm-allocator}
\end{figure}

\subsection{数据库元数据区的结构设计}

为了保证 N2DB 在故障恢复的过程中找到所有的数据库中所有的元数据。因此 N2DB 需要增加一个元数据区负责存储和管理所有数据库的元信息。元数据区的结构如图~\ref{fig:catalog} 所示。元数据区中的数据结构组织成一颗三层高的树,其根节点的指针存储在 NVM 分配器的指针区域中。元数据区根据树的层次分别由三类数据结构,从根节点到叶节点分别是 N2DB 元数据,数据库元数据以及表格元数据。

N2DB 元数据中记录的整个存储引擎的元数据,比如存储引擎是否被初始化,这一轮的事务 ID 的起始值等信息。一个存储引擎可以创建多个数据库,因此 N2DB 元数据中还存放着数据库元数据的数量有以及各个数据库元数据的地址。

数据库元数据用于存储数据库的元数据,其具体的数据结构定义如下:
\begin{itemize}
    \item 初始化标志(is\_initialized):该标志用于判断该元数据是否被初始过。数据库在创建的过程中会先设置其他的成员变量的数据,最后原子化设置该标志,如果该标志不为 1,则数据库视为创建失败。
    \item 数据库标识符(database\_id):每个数据库都有全局唯一的数字标识符。
    \item 表格数量(table\_count):该变量记录了从属于数据库的表格的数量。
    \item 表格元数据入口(table\_entries):表格元数据入口是一个定长的数组,其中每一个元素均是表格元数据的指针。由于该数组仅存储一种类型的数据结构的指针,因此指针不需要附带类型信息。
\end{itemize}

表格元数据用于存储表格的所有元数据,包括表格的标识符等信息。同时表格元信息中也存储了模式信息,数据库管理系统可以借此计算出表格中所有的 head 和 version 的大小。表格元信息还存储了表格所使用的所有页面信息。表格元数据具体的数据结构定义如下:
\begin{itemize}
    \item 初始化标志(is\_initialized):该标志用于判断该元数据是否被初始过。与数据库元数据类似,表格元数据根据此标志判断表格是否创建成功。
    \item 表格标识符(table\_id):每个表格的全局唯一的数字标识符。
    \item 列数(col\_count):列数记录了表格的列的数量。
    \item 列信息数组(col\_infos):列信息数组用于记录表格的模式信息。数组中每一个列信息均是一个三元组,分别记录了列名称,列大小以及列类型。
    \item head 页面入口(head\_pages\_entris):head 页面入口是一个数组,其中每一个元素均是一个页面指针。每个页面都是专门用来存放表格的 head。
    \item version 页面入口(version\_pages\_entries):version 页面入口同样也是一个页面指针数组。其中的每个页面都是专门用来存放表格的 version。
\end{itemize}

\begin{figure}
    \centering
    \includegraphics[width=0.6\linewidth]{example-image-a.pdf}
    \caption{数据库元信息区的结构}
    \label{fig:catalog}
\end{figure}

\section{数据恢复流程}

我们所涉及的恢复策略有以下几个目标。首先是 NVM 空间的管理系统的重建。一些 NVM 工作使用的是类似于 PMDK 的 NVM 分配器协助管理,其好处在于使用者不需要手动维护分配信息,但是其代价在于 PMDK 是通过日志保证分配信息的持久化的,因此延迟较大,通常有(我记得某篇论文里有PMDK 的分配延迟)。因此我们采用的是手动管理 NVM 空间,因此在重启之后,需要根据持久化信息重新得到空间的分配信息。其次是数据库管理系统的重建,根据 NVM 空间管理系统的分配信息,我们可以找到数据库管理系统的元数据。根据这些信息,系统需要恢复事务,表格,索引等等信息。最后则是数据完整性,在我们上文提到过,数据的完整性主要通过遵循两个原则来保证。在重启之后,数据库管理系统会得到所有的正确的表格信息,在此基础上,系统需要保证提交的事务的影响以及消除中止和因宕机停止的事务的影响,回收空余的空间,并且尽可能快地提供后续的正常的服务。


\subsection{NVM 分配器的恢复流程}

\subsection{数据库元数据的恢复流程}

\subsection{记录数据的恢复流程}

\section{数据恢复的正确性}


\section{系统工程实现}

\section{本章小结}